% !TEX root = ../my-thesis.tex
%
\chapter{Introduction}
\label{sec:intro}

In the last century, particle physics has achieved a description of Nature at microscopic scales vastly exceeding the ``small distances as hitherto escape Observation'' whose exploration Isaac Newton foreshadowed~\cite{Newton1704}. The best theory in terms of accurately and fundamentally describing the observed phenomena to date is the Standard Model (SM) of electroweak and strong interactions. The SM is confirmed by a series of breakthrough discoveries~\cite{Pais1988}, including the observation of the \PW and \PZ bosons in the year 1983~\cite{Arnison1983,Banner1983,Arnison1983-2,Bagnaia1983} and the discovery of the Higgs boson in the year 2012~\cite{HIGG-2012-27,CMS-HIG-12-028}, which completed the inventory of the Standard Model's particle content.

At the same time, astrophysical observations at macroscopic scales indicate the presence of gravitationally interacting but otherwise invisible non-baryonic matter in the universe~\cite{Bertone2010}. Despite the tremendous success of the SM, it does not provide a particle candidate for the so-called dark matter. Elucidating on its potential particle nature, its origin and its interactions is among the most important problems in fundamental physics. Dark matter particles may be produced in high-energy proton-proton collisions at the Large Hadron Collider (LHC). If produced, dark matter remains elusive to the detectors and may only be detected via its recoil against other particles.  The very same particles, whose discovery corroborated the SM are now used as probes for dark matter production.

This dissertation discusses four searches for dark matter at the LHC with the ATLAS detector.

First, a search for dark matter in association with a hadronically decaying weak vector boson is discussed. The hadronic decay products of the vector boson give rise to collimated sprays of particles, which are referred to as jets. Depending on the Lorentz boost of the weak vector boson candidate, it is reconstructed either using two well-separated jets or a single large-radius jet with jet substructure information.

Second, a search for dark matter production in association with a Higgs boson decaying to \bquarks is discussed.
The Higgs boson candidate is reconstructed from the striking signature of two \bjets in the detector, which are identified using multivariate techniques, known as \btagging. In events with boosted Higgs boson candidates, these are reconstructed using large-radius jets, while the \btagging information is supplemented by sub-jets constructed from inner detector tracks. High \btagging efficiency in event topologies with highly boosted Higgs boson candidates is achieved by reconstructing the sub-jets with a variable radius size which adapts to the momentum of the Higgs boson candidate.

Third, a reinterpretation of the previously discussed search in terms of a hypothetical dark Higgs boson decaying to a pair of \bquarks is discussed. The existence of a dark Higgs boson is motivated by spontaneous symmetry breaking in the dark sector. The dark Higgs boson can be sought for in its decays into visible particles due to mixing with the SM Higgs boson. The reinterpretation is enabled by the RECAST framework and represents the proof-of-concept of faithful and automated reinterpretation of ATLAS dark matter searches.

Finally, a search for dark matter production in association with a dark Higgs boson decaying to a pair of weak vector bosons in the hadronic decay channel is discussed.
The challenging event topology of up to four jets from diboson decay requires the use of a novel jet reconstruction technique. Track-assisted-reclustered (TAR) jets with large radius parameter are formed by reclustering small-radius jets. The jet substructure information is computed from inner detector tracks which are matched to the small-radius jets. Two event topologies are considered, in which the TAR jet either contains the full dark Higgs decay or is complemented by adding additional small-radius jets.

No significant deviations from the SM background predictions are observed in the searches. They are interpreted in terms of simplified models of dark matter production with varying degrees of complexity to set limits on the parameter space of these models.

In addition to the searches for dark matter, a study about the optimisation of the first-level ATLAS muon trigger is presented. Modifications of the trigger coincidence logic enable a reduction of the trigger rate in the forward region of the detector while maintaining a high trigger efficiency.

\vspace{5.5cm}
\textbf{Personal contributions}

High-energy physics experiments are conducted in extensive, international collaborations. The searches for dark matter presented in this dissertation have been performed using data recorded by the ATLAS experiment, which has been designed, constructed, and maintained by an international collaboration of more than \num{3000} persons. Most tasks, such as detector maintenance and operation, development and calibration of event reconstruction algorithms, and distributed analysis of the recorded data is carried out centrally in dedicated working groups.
Therefore, this work builds on the contribution of many past and present members of the ATLAS collaboration, whose contributions are referenced throughout the text.
Figures with the label \textbf{ATLAS} have been shown in ATLAS peer-reviewed publications, while those with the label \textbf{ATLAS Preliminary} have been featured in ATLAS conference notes or public notes.
Figures without those labels have been taken from non-ATLAS publications and are referenced accordingly.
All figures and tables without such a reference in the caption have been produced by the author of this dissertation.

The results presented in this dissertation have been published as peer-reviewed journal publications and as ATLAS conference notes or ATLAS public notes, which have undergone a multi-stage internal peer-review process by the ATLAS collaboration. The contributions of the author in these endeavours are listed below.
\begin{itemize}
	\item \(\met + \Vqq\) search in Ref.~\cite{EXOT-2016-23} and in Ref.~\cite{EXOT-2017-32}\\
	      \emph{The author made large contributions to the analysis software development and produced the inputs on which the statistical analysis is based. He developed and applied the multijet background estimation technique, which was also used in Ref.~\cite{EXOT-2016-25}. The author was responsible for the derivation of the exclusion limits on models of dark matter production and carried out all studies on validating the statistical model. He further served as a liaison for incorporating the results in a publication summarising the broad ATLAS dark matter and dark energy research programme. Finally, the author was co-editor of the internal documentation.}
	\item \(\met + \Hbb\) search in Ref.~\cite{ATLAS-CONF-2018-039}\\
	      \emph{The author maintained the analysis software and made large contributions to it. He commissioned a novel object-based \met significance observable and studied its optimal use in the analysis, including an estimate of its effect in reducing the multijet background. The author produced the inputs on which the statistical analysis is based. He was responsible for the derivation of the exclusion limits on models of dark matter production and carried out all studies on validating the statistical model. Furthermore, he investigated the relative improvement in the limits due to the novel variable-radius track jet algorithm.}
	\item \(\met + \sbb\) reinterpretation in Ref.~\cite{ATL-PHYS-PUB-2019-032}\\
	      \emph{The author implemented the \(\met + \Hbb\) search in the RECAST framework and devised the simulated signal samples. He co-coordinated the effort and co-edited Ref.~\cite{ATL-PHYS-PUB-2019-032} with Lukas Heinrich.}
	\item \(\met + \sVV\) (hadronic) search in Ref.~\cite{ATLAS-CONF-2020-036}\\
	      \emph{The author developed the analysis software using the \textsc{XAMPP} framework and maintained it. He implemented the TAR jet algorithm and the associated systematic uncertainty estimation and validated its performance. The author devised the simulated signal samples and was responsible for managing the datasets of collision data and simulated events. He made large contributions towards designing the event selection. The author produced the inputs on which the statistical analysis is based and implemented the statistical model. He was responsible for the derivation of the exclusion limits on models of dark matter production and carried out all studies on validating the statistical model. Furthermore, he studied the effect of theory systematic uncertainties on signal and background processes. Finally, the author was co-editor of the internal documentation.}
\end{itemize}


\newpage
\mbox{}
\vfill
\textbf{System of units}

Throughout this dissertation, natural units are used. In contrast to the \emph{Système international d'unités} (SI), the units of the most common observables are expressed by natural constants. Formally this is realised by setting the speed of light \(c\), Planck's constant \(\hbar\), and the electric field constant \(\varepsilon_{0}\) to unity
\begin{align}
    c = \hbar = \varepsilon_{0} = 1.
\end{align}
The unit of energy is not specified and is chosen to be the electron volt (\([E] = \si{\electronvolt}\)), which is the amount of kinetic energy a point charge of \SI{1}{\coulomb} gains from acceleration in an electric field with potential difference of \SI{1}{\volt}. Consequently, all units are expressed in powers of \si{\electronvolt}.
As gravity usually is neglected in the description of sub-atomic phenomena, Newton's constant is still expressed in SI units.
