% !TEX root = ../my-thesis.tex
%
\pdfbookmark[0]{Acknowledgements}{Acknowledgements}
\addchap*{Acknowledgements}
\label{sec:acknowledgement}

As every good story comes to an end, so does the scientific narrative of this dissertation.
Now it is time to let the protagonist of the story --- the elusive dark matter --- step back into darkness and leave the stage to all the people who contributed in one way or another to this dissertation. There are many people whom I would like to thank in particular.

First of all, I want to thank Oliver Kortner for supervising my PhD. Already at the time of my master thesis, your enthusiasm and guidance let me grow personally and as a scientist. You taught me to find something that interests me and fully engage in it. I am thankful for your trust and your support in giving me the opportunity of pursuing my research interests.
No less, I want to thank Sandra Kortner for all our insightful discussions, always keeping an open door, and supporting my work in every conceivable way. Further, I am thankful for providing many opportunities to visit summer schools, present my research at conferences, and enabling frequent visits to CERN.

Thank you to Peter Fierlinger and Norbert Kaiser for agreeing to be on my dissertation committee as a second examiner and the head of the committee, respectively.

I wish to express my gratitude towards Siegfried Bethke, Hubert Kroha, and the International Max Planck Research School on Elementary Particle Physics under the coordination of Frank Steffen for providing a productive environment for my research at the Max-Planck-Institute for Physics in Munich.

Patrick Rieck's support deserves special mention. Our collaboration was most enjoyable and formative, thanks to your no-nonsense approach to physics analysis.
Your supervision provided an ideal transition from an academic apprenticeship in my first days towards learning from your constructive criticism of my later work in your role as a JDM subgroup convener. Furthermore, I thank you for reading and providing comments on selected chapters of this dissertation.

The field of high-energy physics is stimulating not only because of the research itself but also because of getting to collaborate with talented and engaging colleagues.
I want to thank Oleg Brandt for his mentoring and thank him and Christopher Lester for the hospitality granted at the Cavendish Laboratory. Equally, I am thankful for the chats about physics and the advice from Shih-Chieh Hsu.
Sam Meehan's honest and unbounded enthusiasm about physics and so many other things lifted my spirit more than once and set an example to look up to.

The work presented in this dissertation was conducted in collaboration with many other researchers.
I thank Patrick Rieck, Jike Wang, Katharina Behr, Helene Genest, Frank Filthaut, Shih-Chieh Hsu, Stanislava Sevova, and Oleg Brandt for their work as analysis contacts in coordinating the searches for dark matter presented in this thesis.
During my work on these searches, I encountered many persons who taught me about physics analysis and with whom I engaged in joint research. In particular, I enjoyed the collaboration with %
% trigger
Junpei Maeda, %
% \met + \Vqq
Xuanhong Lou, Lailin Xu, Kevin Bauer, Krisztian Peters, Koji Terashi, %
% \met + \Hbb
Andrea Matic, Veronica Fabiani, Dilia Portillo, Ruth Pöttgen, Pai-hsien Jennifer Hsu, %
% \met + \sbb
Lukas Heinrich, Fabrizio Napolitano, %
% \met + \sVV
Stany Sevova, Danika MacDonell, Cong-qiao Li, Flavia Dias, Jason Veatch, Philipp Mogg, Jeanette Lorenz, Lauren Tompkins, Ning Zhou, %
% \met + \Hbb full run 2
Jay Chan, Anindya Ghosh, Jon Burr, Dan Guest, and Spyridon Argyropoulos.

The days at the Max-Planck-Institute were filled with life by Max-Goblirsch-Kolb, Felix Müller, Natascha Savic, Nicolas Köhler, Tom McCarthy, Stefan Maschek, Dominik Duda, Davide Cieri, Nina Wenke, and Michael Holzbock. My wonderful office-mates Katharina Ecker, Verena Walbrecht, Andreas Hönle, and frequent guest Johannes Junggeburth not only endured my moods which directly reflected on the loudness of my keyboard but also provided the best company I could wish for.

My friends Matthias Mader, Andreas Rauscher, Agnes Köhler, Christopher Mittag, Vitaly Wirthl, Christoph Manß, Jan-Thorge Schindler, Daniel Rosenblüh, Wolfgang Neumeyer, Felix Monninger and Sarah Park provided me with welcome distraction and continue to remind me that there is more to life than doing physics.

Lastly, I want to thank my entire family, all aunts, uncles, and cousins who took an interest in what I was doing over the last four years.
In particular, my aunt Waltraud, whose literacy knows no bounds, revealed the most beautiful books to me and kindled my interest in science.
All this work would not have been possible without the loving and unwavering support of my parents Karin and Richard. I cannot thank you enough for everything you have done for me. You taught me to never stop learning, never give up quickly, and follow my interests.

Finally, I want to thank you, Denise, for your kindness, your patience, and your love. I am glad to have you in my life.
