% !TEX root = ../my-thesis.tex
%
\pdfbookmark[0]{Abstract}{Abstract}
\addchap*{Abstract}
\label{sec:abstract-english}
\vspace{-4mm}
Dark matter is a form of invisible matter which is predicted to constitute a vast portion of the universe. Although its existence is corroborated by numerous astrophysical observations on different cosmological scales, its direct observation is still pending, and its particle nature is an open question.
This thesis describes searches for the production of dark matter particles using proton-proton collision data recorded by the ATLAS detector at the Large Hadron Collider at \(\sqrt{s} = \SI{13}{\tera\electronvolt}\) centre-of-mass energy, investigating signatures of missing transverse momentum and hadronically decaying bosons.
No significant deviations from the Standard Model prediction are observed, thereby providing constraints on the parameter space of models describing mediator-based dark matter production.

The \(\met + \Vqq\) search targets processes with substantial missing transverse momentum \met and hadronically decaying weak vector bosons \(V\). It is based on a data sample corresponding to an integrated luminosity of \SI{36.1}{\per\femto\barn}.
The results of the \(\met + \Vqq\) search are interpreted in terms of a spin-1 \(Z'\) mediator simplified model, excluding \(Z'\) mediator masses of up to \SI{830}{\giga\electronvolt} for dark matter masses up to \SI{280}{\giga\electronvolt} at \SI{95}{\percent} confidence level for mediator couplings to quarks of \num{0.25} and dark matter particles of \num{1}. The results are also interpreted in terms of a simplified model with an extended Higgs sector and a pseudo-scalar mediator.

The \(\met + \Hbb\) search investigates a similar experimental signature, targeting Higgs bosons \(h\) decaying to \(b\)-quarks. It is based on a data sample corresponding to an integrated luminosity of \SI{79.8}{\per\femto\barn} and exploits a jet algorithm with a variable radius parameter for the reconstruction of Higgs boson candidates in boosted event topologies. The results are interpreted in terms of a \(Z'\)-2HDM simplified model. For a specific choice of the model parameters, masses of the \(Z'\) boson are excluded up to \SI{2.85}{\tera\electronvolt} at \SI{95}{\percent} confidence level.

The \(\met + \sbb\) and \(\met + \sVV\) searches investigate the yet uncharted signature of missing transverse momentum and the production of a hypothetical dark Higgs \(s\) decaying to \(b\)-quarks or pairs of weak vector bosons in their hadronic decay mode. While the \(\met + \sbb\) signature is investigated by a reinterpretation of the \(\met + \Hbb\) search using the RECAST framework, the \(\met + \sVV\) search is based on the full Run-2 dataset corresponding to an integrated luminosity of \SI{139}{\per\femto\barn}. The reconstruction of the dark Higgs boson candidates is based on a novel jet reconstruction algorithm, which combines the calorimeter and inner detector tracking information. The results are interpreted in terms of a simplified model with spin-1 \(Z'\) boson and spin-0 dark Higgs boson mediators, excluding dark Higgs boson masses of up to \SI{230}{\giga\electronvolt} and \(Z'\) boson masses of up to \SI{1.8}{\tera\electronvolt} at \SI{95}{\percent} confidence level for a specific choice of the other model parameters.

The implications of these results are discussed in the context of a summary of ATLAS dark matter searches.
