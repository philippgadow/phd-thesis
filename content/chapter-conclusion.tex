% !TEX root = ../my-thesis.tex
%
\chapter{Conclusion}
\label{sec:conclusion}
% topic of the thesis
The investigation of the particle nature of dark matter is among the central research problems of the LHC physics programme. The unprecedented Run-2 \HepProcess{\Pp\Pp} collision dataset collected at a centre-of-mass energy of \SI{13}{\tera\electronvolt} enables searches for associated dark matter production in various final states and event topologies. This thesis investigates the production of dark matter in signatures of missing transverse momentum \met and hadronically decaying bosons.

% trigger
The MC-based background estimation in these searches is improved by auxiliary measurements in control regions, which rely on efficient identification of leptons. As these measurements are ultimately limited by the trigger, improvements in the trigger selectivity directly impact the recorded data. An \textbf{optimisation of the low-\pt L1 muon trigger coincidence logic} results in a reduction of the L1MU4 trigger rate of \SI{20}{\percent} while maintaining a muon efficiency of \SI{99}{\percent} in the forward region of the detector in the range \(2.0 < \abs{\eta} < 2.4\).

% dark matter searches
In addition to the trigger optimisation study, four \textbf{searches for dark matter produced in association with hadronically decaying heavy bosons} are presented in this dissertation. In all searches, the signature of the signal process is \(\met\) arising from the elusive dark matter particles and jets from the heavy boson decay.
The searches are motivated by simplified models for mediator-based dark matter production with varying degrees of complexity, which are used for the interpretation of the results by expressing constraints on the parameter space of these models.

% \met + \Vqq
The \textbf{\(\met + \Vqq\) search} investigates signatures with weak vector boson candidates, which are reconstructed either as a single large-radius jet if they are boosted or as a pair of small-radius jets otherwise. The weak vector boson candidates are required to be compatible with the \PW and \PZ mass and jet substructure of their hadronic decays.
The ATLAS \HepProcess{\Pp\Pp} collision data recorded in the years 2015--2016, corresponding to an integrated luminosity of \SI{36.1}{\per\femto\barn} are analysed by selecting events in categories defined by the event topology and the \bjet multiplicity. The backgrounds estimate based on MC simulation is improved by the use of control region measurements for the dominant background processes \zjets, \wjets, and \ttbar production.
A combined likelihood-based statistical analysis of all regions is performed with \met as the discriminating variable.
Two simplified models, a spin-1 \PZprime boson mediator model and a model with extended Higgs sector and a pseudo-scalar mediator, are probed by comparing the hypothesis of dark matter signals on top of the SM backgrounds against the null hypothesis. The statistical model takes into account various sources of experimental and theoretical systematic uncertainty.
No significant excess over the SM prediction is observed. Therefore, exclusion limits on dark matter production in the simplified models under consideration are derived for model parameters recommended by the LHC DM WG.
In the spin-1 \PZprime boson mediator model, limits are set in the \mZp-\mchi plane with fixed mediator couplings \(\gq = 0.25\), \(\gchi = 0.25\), and \(\gl = 0\). Dark matter production via the \(s\)-channel exchange of a \PZprime boson with vector couplings is excluded at \SI{95}{\percent} \(\text{CL}_{s}\) for \mZp up to \SI{830}{\giga\electronvolt} and \mchi up to \SI{280}{\giga\electronvolt}. The limits on \PZprime boson mediators with axial-vector couplings have similar reach in \mZp but cover a smaller region of the on-shell region.
Only the \(\met + \PZ(\Pq\Pq)\) signature is relevant for setting limits on the simplified model with an extended Higgs sector and a pseudo-scalar mediator, which is therefore only mildly constrained. For the parameter choices \(\tan{\beta} = 1.0\), \(\mchi = \SI{10}{\giga\electronvolt}\), and \(\sin \theta = 0.35\), configurations with heavy Higgs boson masses \mH in the range \SIrange{800}{1050}{\giga\electronvolt} and pseudo-scalar mediator masses of up to \SI{200}{\giga\electronvolt} are excluded at \SI{95}{\percent} \(\text{CL}_{s}\). Similarly, values of \(\sin \theta > 0.5\) are excluded for a model configuration with \(\mHiggsHeavy = \SI{600}{\giga\electronvolt}\), \(\ma = \SI{200}{\giga\electronvolt}\), \(\tan{\beta} = 1.0\), and \(\mchi = \SI{10}{\giga\electronvolt}\).

% \met + \Hbb
The \textbf{\(\met + \Hbb\) search} investigates signatures with Higgs boson candidates in the \bbbar final state, which are reconstructed as a single large-radius jet with two \btagged sub-jets or as a pair of \bjets. The search is sensitive to highly boosted Higgs boson candidates by employing a track-based sub-jet reconstruction algorithm with a variable radius size, which adapts to the momentum of the Higgs boson candidate (VR track jets).
The ATLAS \HepProcess{\Pp\Pp} collision data recorded in the years 2015--2017, corresponding to an integrated luminosity of \SI{79.8}{\per\femto\barn} is analysed by selecting events in four \met bins. The background estimate of the dominant \ttbar, \zjets and \wjets backgrounds is improved by dedicated control regions. The combined likelihood-based statistical analysis, which is based on the Higgs candidate mass as the primary discriminating variable, probes a simplified model with an extended Higgs sector and a \PZprime mediator.
As no significant excess over the SM background is observed, exclusion limits on the simplified model are derived, which exclude large regions in the two-dimensional \mZp-\mA plane for the fixed set of parameters  \(g_{\PZprime} = 0.8\), \(\tan \beta = 1\), \(\mchi = \SI{100}{\giga\electronvolt}\), and \(\mHiggsHeavy = \mHiggsCharged = \SI{300}{\giga\electronvolt}\) with \mZp of up to \SI{2.85}{\tera\electronvolt} and \mA up to \SI{670}{\giga\electronvolt}.
The model, which is already strongly constrained by dijet searches, is chosen as a benchmark to compare the performance of the novel VR track technique to a previous iteration of the search. For configurations with \(\mZp > \SI{2.5}{\tera\electronvolt}\), which correspond to event topologies with highly boosted Higgs candidates, the performance is improved up to a factor of 3, thereby establishing the advantage of VR track jets in future boosted resonances searches.

% \met + \sbb
The \textbf{\(\met + \sbb\) search} investigates signatures with a hypothetical dark Higgs boson decaying to \bquarks. As the final state of \met and \bjets is shared with the \(\met + \Hbb\) search, the latter is reinterpreted using the RECAST framework to place limits on a simplified model with a spin-1 \PZprime boson mediator and a spin-0 dark Higgs boson mediator. The RECAST framework enables faithful and automated reinterpretations by preserving not only the observed data and the background model of the original search but also the analysis software and the full analysis workflow.
The reinterpretation places stringent limits on the model under consideration for the fixed choice of parameters \(\mchi = \SI{200}{\giga\electronvolt}\), \(\gq = 0.25\), and \(\gchi = 1.0\) by excluding the configurations compatible with relic density measurements in the region \(\SI{50}{\giga\electronvolt} < \ms < \SI{150}{\giga\electronvolt}\).
The search demonstrates the feasibility of RECAST-based reinterpretations, which are of paramount importance for exploiting the full potential of searches given the increasing number of models being proposed. RECAST is becoming an integral part in the preservation efforts of full Run-2 ATLAS searches.

% \met + \sVV
The \textbf{\(\met + \sVV\) search} investigates signatures with a hypothetical dark Higgs boson decaying to pairs of weak vector boson in their hadronic decay mode. The novel Track-Assisted-Reclustering (TAR) jet algorithm is employed to reconstruct the collimated \sVV system, using small-radius jets augmented with precision tracking information to form TAR jets with jet substructure resolution superior to conventional jet reconstruction techniques.
The dark Higgs boson candidate is reconstructed either as a single TAR jet or as a TAR jet supplemented with one or two small-radius jets in event topologies with less boosted dark Higgs boson candidates.
The search is based on the full Run-2 ATLAS \HepProcess{\Pp\Pp} collision data recorded in the years 2015--2018, corresponding to an integrated luminosity of \SI{139}{\per\femto\barn}. Again, control regions improve the MC-based background estimate of the dominant background processes \zjets and \wjets production.
The statistical analysis is performed using the dark Higgs candidate mass as the discriminating variable in three categories, which are based on the event topology and \met.
The simplified model with two mediators which is also probed by the \(\met + \sbb\) search is considered for the interpretation of the results, focusing on dark Higgs masses \(\ms > \SI{160}{\giga\electronvolt}\) where decays to weak vector boson pairs become relevant.
A mild localised excess in data is observed, which corresponds to a local significance of \(2.3\sigma\) for the dark Higgs boson hypothesis with \(\ms = \SI{160}{\giga\electronvolt}\) and a global significance of \(1.3\sigma\) when considering the nine independent \ms hypotheses under consideration. As no large deviations from the SM background are observed, limits for the fixed choice of parameters \(\mchi = \SI{200}{\giga\electronvolt}\), \(\gq = 0.25\), and \(\gchi = 1.0\) are computed in the \mZp-\ms plane.
Signal configurations with \mZp of up to \SI{1.8}{\tera\electronvolt} and \ms in the range \SIrange{170}{230}{\giga\electronvolt} are excluded at \SI{95}{\percent} \(\text{CL}_{s}\).

% final thoughts
Although no significant evidence of dark matter production was observed in these searches, these null results fit in the larger \textbf{puzzle of dark matter}, whose pieces are unveiled by the combined efforts of experiments probing its microscopic and macroscopic interactions and by theorists exploring the viable models of dark matter.
The searches for dark matter at particle colliders --- windows on the microscopic interactions of dark matter -- are an integral part of solving the dark matter puzzle. The ATLAS detector will continue to collect \HepProcess{\Pp\Pp} collision data in the upcoming LHC runs over the next decade, eventually increasing the size of the dataset by order of magnitude.
The novel reconstruction techniques employed in the searches presented in this dissertation improve the sensitivity to boosted resonances in dense environments and enable the exploration of yet uncovered signatures.
As the increasingly involved reconstruction techniques render these searches more complex and challenging, an automated framework enabling faithful reinterpretations like RECAST is required to exploit the dataset sustainably and fully.
The search for dark matter is still ongoing, leaving the scientific community in shared wonder about the true nature of dark matter. May the results of this work benefit future searches for new phenomena and testify the joy of curious enquiry.
